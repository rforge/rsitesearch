\title{Searching help pages of R packages}
\author{by Spencer Graves, Sundar Dorai-Raj, and Romain Francois}

\maketitle

The {\tt sos} package provides a means to quickly and
flexibly search the help pages of contributed packages, finding
functions and datasets in seconds or minutes that could not be
found in hours or days by any other means we know.

The main capability of this package is the {\tt findFn} function,
which queries only the "function" help pages in Jonathan Baron's
RSiteSearch data base and returns the results in a {\tt data.frame}
of class {\tt findFn}.  The corresponding {\tt print} method
displays the results as a table in a web browser with links to the
individual help pages, sorted by package displaying the one with the
most matches first.  Other {\tt sos} functions provide a summary
with one line for each package, support the union and intersection
of {\tt findFn} objects, and allows the results to be
written to an Excel file with three sheets for (1) {\tt PackageSum2}, which
provides an enhanced summary of the packages with matches, (2) the
{\tt findFn} table itself, and (3) the {\tt call} used to produce it.

Other \R{} functions can then be used to
quickly find what you want among possibly hundreds or thousands of
hits or produce

Two examples are considered below:  First we find a dataset containing
a variable {\tt Petal.Length}, used without indicating the source 
by Chambers (2009, pp. 282-3).  Second, we study \R{} capabilities 
for splines, including looking for a function named {\tt spline}.

\section*{Petal.Length}

Chambers (2009, p. 282-283) uses a variable {\tt Petal.Length} from a 
famous Fisher data set but without naming the dataset nor indicating 
where it can be found nor even if it exists in \R{}.  The sample 
code he provides does not work by itself.  To reproduce his 
Figure 7.2, we must first obtain a copy of this famous data set 
in a format compatible with Chambers' code.  

How to add Bibliography ?
Chambers, John (2009) Software for Data Analysis (Springer) 

Some users might try the following:  
\begin{verbatim}
> help.search('Petal.Length')
No help files found ...
\end{verbatim}
When this failed, many users might then try 
{\tt RSiteSearch('Petal.Length')}.  This produced 80 hits.  
{\tt RSiteSearch('Petal.Length', 'function')} will identify 
only the help pages on this list, but we can get something 
similar and more useful as follows:
\begin{verbatim}
> library(sos) 
> PL <- findFn('Petal.Length')
\end{verbatim}
{\tt PL} is a {\tt data.frame} of class {\tt findFn} identifying 
all the help pages in Jonathan Baron's data base matching the 
search term.  

The {\tt summary} method for such an object returns 
the number of matches with a table giving for each {\tt Package} 
the {\tt Count} (number of matches), {\tt MaxScore} (max of the 
{\tt Score}), {\tt TotalScore} (sum of {\tt Score}), and {\tt Date}, 
sorted like a Pareto chart to place the {\tt Package} with the most 
help pages first:  
\begin{verbatim}
> summary(PL)

Total number of hits: 23
Number of links downloaded: 23

Packages with at least 1 hit
using search pattern 'Petal.Length':
          Count MaxScore TotalScore
yaImpute      8        1          8
<...>
datasets      1        2          2
<...>
\end{verbatim}

REDO to add the Date  

One of the listed packages is {\tt datasets}.  Since it's part of the
default \R{} distribution, we decide to look there first.  We can
select that row of PL just like we would select a row from any other
data.frame:
\begin{verbatim}
> PL[PL$Package=='datasets', 'Function']
[1] iris
\end{verbatim}
The {\tt print} method for an object of class {\tt findFn} 
opens the result in a browser with the last column being linked 
to the associated help page.  

Problem solved in less than a minute!  Any other method known 
to the present authors would have taken substantially more time.  

\section*{spline}

Three years ago, I decided I wanted to learn more about
splines.  I started my literature search as follows:
\begin{verbatim}
RSiteSearch('spline')
\end{verbatim}
While preparing this manuscript, this command identified 1526
documents.  That is too much, so I restricted it to functions:
\begin{verbatim}
RSiteSearch('spline', 'fun')
\end{verbatim}
This identified only 631.  That's an improvement over 1526 but is 
still too much.  To get a quick overview of these 631, we can 
proceed as follows:
\begin{verbatim}
splinePacs <- findFn('spline')
\end{verbatim}
This downloaded a summary of the 400 highest-scoring help pages in
the 'RSiteSearch' data base in roughly 5-15 seconds, depending on the
speed of the Internet connection.  To get all 631 hits, increase
{\tt maxPages}:
\begin{verbatim}
splineAll <- findFn('spline', maxPages=999)
\end{verbatim}
As noted above, the {\tt print} method will open the result in 
a web browser.  

However, a table with 631 rows is rather large to digest easily.  
We could try the {\\ summary} method, but that produces a table 
with 

HOW MANY?  

rows.  The simplest thing to do from here is to create an Excel 
file as follows:  
\begin{verbatim} 
writeFindFn2xls(splineAll) 
\end{verbatim} 
This produces an Excel file (which can be opened with Open Office 
Calc 

citation for Open Office Calc?

asdf, which can be useful for people who do not have Excel) with 
three sheets:  

To find a function named {\tt spline} from this, we can proceed as
follows:
\begin{verbatim}
selSpl <- (splineAll[,'Function']=='spline')
splineAll[selSpl, ]
\end{verbatim}
This has 0 rows, because there is no help page named {\tt spline}.

We can expand this to include any help page containing {\tt spline} in
the name using {\tt grepFn}:
\begin{verbatim}
> grepFn('spline', splineAll, ignore.case=TRUE) 
\end{verbatim}
This returned a {\tt findFn} object identifying 66 help pages.  
The {\tt print} method for an object of class {\tt findFn} 
presents the result in a web browser, 

asdf, 
the first of which is 'lspline' in the
'assist' package.  The {\tt RSiteSearch} engine assigned it a {\tt
Score} of 1.  Evidently, that search engine found only minimal
evidence of its relevance to the requested search {\tt string}.  It
appeared at the top of this list, because the {\tt assist} package had
34 help pages identifed as potentially relevant to that search {\tt
string}, none of which had a {\tt Score} exceeding 1.  

To establish priorities among different packages for further study, it
might be nice to have a Pareto chart showing the 10 packages with the 
most help pages relevant to our search {\tt string}.  We can get this as
follows:
\begin{verbatim}
> spSm <- attr(splineAll,'PackageSummary')
> spSm[1:10,'Count']
    assist     fda           gss      mgcv
        34      30            25        22
      VGAM kernlab DierckxSpline bayesSurv
        17      17            16        16
smoothSurv splines
        15      14
\end{verbatim}
To obtain a similar Pareto by 'TotalScore' requires a little more
effort:
\begin{verbatim}
> o <- rev(order(spSm[, 'TotalScore']))
> splineSum[o, ][1:10, ]
              Count MaxScore TotalScore
gss              25       35        448
splines          14       45        354
fda              30       48        275
<...>
\end{verbatim}
This analysis gave us in seconds a very informative overview of {\tt
spline} capabilities in contributed \R{} packages in a way that can help
establish priorities for further study of the different packages and
functions.

\section*{HTML}
The {\tt HTML} function writes an {\tt RSiteSearch} object to a file
in HTML format and opens it in a browser from which a mouse click will
open a desired help page.

The power of this can be seen by applying this function to the
{\tt grep'ed} subset of help pages with names including the phrase
{\tt spline}:

\begin{verbatim}
HTML(splineAll[select, ])
\end{verbatim}

Of the 631 help pages containing {\tt spline}, this displayed only
those whose name included the phrase {\tt spline}.  Similar analyses
could display any desired subset of an {\tt RSiteSearch} object
created from merging several calls to {\tt RSiteSearch.function}.

\section*{Summary}
In sum, we have found {\tt RSiteSearch.function} in the {\tt
RSiteSearch} package to be a very quick and efficient method for
finding things in contributed packages.

\section*{Acknowledgments}
The {\tt RSiteSearch} capabilities here extend the power of the
{\tt RSiteSearch} search engine maintained by Jonathan Baron.
Without Prof. Baron's support, it would not have been feasible
to develop the features described here.
\newline \newline
\emph{Spencer Graves \newline
Productive Systems Engineering \newline
San Jose, CA \newline
email:  {\tt spencer.graves@prodsyse.com} }
\newline \newline
\emph{Sundar Dorai-Raj \newline
Google \newline
Mountain View, CA \newline
email:  {\tt sdorairaj@google.com} }
\newline \newline
\emph{Romain Francois \newline
\newline
\newline
email:  {\tt romain.francois@dbmail.com} }

