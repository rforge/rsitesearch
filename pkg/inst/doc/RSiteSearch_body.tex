\title{Searching R Packages}
\author{by Spencer Graves and Sundar Dorai-Raj}

\maketitle

The {\tt RSiteSearch} package provides a means to quickly and
flexibly search the help pages of contributed packages, finding
functions and datasets in seconds or minutes that could not be
found in hours or days by any other means we know.

The results are returned in a {\tt data.frame} of class {\tt
RSiteSearch}.  Other \R{} functions can then be used to quickly find
what you want among possibly hundreds or thousands of hits.

Two examples are considered below:  First we find a dataset containing
a variable {\tt Petal.Length}.  Second, we find packages with {\tt
  spline} capabilities, including looking for a function named {\tt
  spline}.

\section*{Petal.Length}

For example, a document discussing \R{} provides an example using a
variable {\tt Petal.Length} from a famous Fisher data set but without
naming the dataset nor where it can be found nor even if it exists in
\R{}.
\begin{verbatim}
> help.search('Petal.Length')
No help files found ...
\end{verbatim}
{\tt RSiteSearch('Petal.Length')} produced 80 hits.
{\tt RSiteSearch('Petal.Length', 'function')} will identify only
the help pages on this list, but we can get the same thing as a
{\tt data.frame} as follows:
\begin{verbatim}
> PL <- RSiteSearch.function('Petal.Length')
\end{verbatim}
The {\tt summary.RSiteSearch} method returns the number of hits, {\tt
max(Score)}, and {\tt sum(Score)} by {\tt Package}:
\begin{verbatim}
> summary(PL)

Total number of hits: 23
Number of links downloaded: 23

Packages with at least 1 hit
using search pattern 'Petal.Length':
          Count MaxScore TotalScore
yaImpute      8        1          8
<...>
datasets      1        2          2
<...>
\end{verbatim}
One of the listed packages is {\tt datasets}.  Since it's part of the
default \R{} distribution, we decide to look there first.  We can
select that row of PL just like we would select a row from any other
data.frame:
\begin{verbatim}
> PL[PL$Package=='datasets', 'Function']
[1] iris
\end{verbatim}
Problem solved in less than a minute!

\section*{spline}

Three years ago, I decided I wanted to learn more about
splines.  I started my literature search as follows:
\begin{verbatim}
RSiteSearch('spline')
\end{verbatim}
While preparing this manuscript, this command identified 1526
documents.  That is too much, so I restricted it to functions:
\begin{verbatim}
RSiteSearch('spline', 'fun')
\end{verbatim}
This identified only 631.  That's an improvement over 1526 but is too
much.  To get a quick overview of these 631, we can proceed as
follows:
\begin{verbatim}
splinePacs <- RSiteSearch.function('spline')
\end{verbatim}
This downloaded a summary of the 200 highest-scoring help pages in
the 'RSiteSearch' data base in roughly 5-10 seconds, depending on the
speed of the Internet connection.  To get all 631 hits, increase
maxPages:
\begin{verbatim}
splineAll <- RSiteSearch.function('spline',
                              maxPages=999)
\end{verbatim}
To find a function named {\tt spline} from this, we can proceed as
follows:
\begin{verbatim}
selSpl <- (splineAll[,'Function']=='spline')
splineAll[selSpl, ]
\end{verbatim}
This has 0 rows, because there is no help page named {\tt spline}.

We can expand this to include any help page containing {\tt spline} in
the name using {\tt grep}:
\begin{verbatim}
> fns <- tolower(splineAll[,'Function'])
> select <- grep('spline', fns)
> splineAll[select, c(1, 4, 5, 7)]
  Count Package             Function Score
31   34  assist              lspline     1
35   30     fda create.bspline.basis    48
<...>
\end{verbatim}
This identified 66 help pages, the first of which is 'lspline' in the
'assist' package.  The {\tt RSiteSearch} engine assigned it a {\tt
Score} of 1.  Evidently, that search engine found only minimal
evidence of its relevance to the requested search {\tt string}.  It
appeared at the top of this list, because the {\tt assist} package had
34 help pages identifed as potentially relevant to that search {\tt
string}.

To establish priorities among different packages for further study, it
might be nice to have a Pareto of the 10 packages with the most help
pages relevant to our search {\tt string}.  We can get this as
follows:
\begin{verbatim}
> spSm <- attr(splineAll,'PackageSummary')
> spSm[1:10,'Count']
    assist     fda           gss      mgcv
        34      30            25        22
      VGAM kernlab DierckxSpline bayesSurv
        17      17            16        16
smoothSurv splines
        15      14
\end{verbatim}
To obtain a similar Pareto by 'TotalScore' requires a little more
effort:
\begin{verbatim}
> o <- rev(order(spSm[, 'TotalScore']))
> splineSum[o, ][1:10, ]
              Count MaxScore TotalScore
gss              25       35        448
splines          14       45        354
fda              30       48        275
<...>
\end{verbatim}
This analysis gave us in seconds a very informative overview of {\tt
spline} capabilities in contributed \R{} packages in a way that can help
establish priorities for further study of the different packages and
functions.

\section*{HTML}
The {\tt HTML} function writes an {\tt RSiteSearch} object to a file
in HTML format and opens it in a browser from which a mouse click will
open a desired help page.

The power of this can be seen by applying this function to the
{\tt grep'ed} subset of help pages with names including the phrase
{\tt spline}:

\begin{verbatim}
HTML(splineAll[select, ])
\end{verbatim}

Of the 631 help pages containing {\tt spline}, this displayed only
those whose name included the phrase {\tt spline}.  Similar analyses
could display any desired subset of an {\tt RSiteSearch} object
created from merging several calls to {\tt RSiteSearch.function}.

\section*{Summary}
In sum, we have found {\tt RSiteSearch.function} in the {\tt
RSiteSearch} package to be a very quick and efficient method for
finding things in contributed packages.

\section*{Acknowledgments}
The {\tt RSiteSearch} capabilities here extend the power of the
{\tt RSiteSearch} search engine maintained by Jonathan Baron.
Without Prof. Baron's support, it would not have been feasible
to develop the features described here.  We also wish to thank
Romain Francois, who had an {\tt RSiteSearch} project on R-Forge
before we did.  He not only agreed to merge his "R Site Search
extension for firefox" project with ours, he also added the
{\tt template} argument to our {\tt HTML} function, thereby
providing added flexibility.
\newline \newline
\emph{Spencer Graves \newline
Productive Systems Engineering \newline
751 Emerson Ct. \newline
San Jose, CA 95126 \newline
email:  {\tt spencer.graves@prodsyse.com} }
\newline \newline
\emph{Sundar Dorai-Raj \newline
Google }
